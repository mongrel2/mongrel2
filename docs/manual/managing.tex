\chapter{Managing}

Mongrel2 is designed to be easy to deploy and \emph{automate} the deployment.
This is why it uses \href{http://www.sqlite.org/}{SQLite} to store the configuration,
but m2sh as an interface to creating the configuration.  Doing this lets
you access the configuration using any language that works for you, augment it,
alter it, migrate it and automate it.

In this chapter, I'm going to show you how to make a basic configuration using
m2sh and all the commands that are available.  You'll learn how the configuration
system is structured so that you know what goes where, but in the end it's just
a simple storage mechanism.

\begin{aside}{Apparently SQL Inspires FUD}
When I first started talking about Mongrel2, I said I'd store the configuration
in SQLite and do a Model-View-Controller kind of design.  Immediately people who
can't read flipped out and thought this meant they'd be back in ``Windows registry hell'',
but with SQL as their only way to access it.  They thought that they'd be stuck writing
configurations with SQL; that SQL couldn't possibly configure a web server.

They were wrong on many levels.  Nobody was \emph{ever} going to make \emph{anyone} use
SQL.  That was repeated over and over but, again, people don't read and love spreading
FUD.  The SQLite config database is nothing like the Windows Registry.  No other web
server really uses a true hierarchy; they just cram a relational model into a weirdo
configuration format.  The real goal was to make a web server that was easy to manage from
\emph{any} language, and then give people a nice tool to get their job done without
having to ever touch SQL.  \emph{EVER!}

In the end, what we got despite all this fear mongering is a bad ass configuration
tool and a design that is simple, elegant, and works fantastic.  If you read that
Mongrel2 uses SQLite and though this was weird, well, welcome to the future.  Sometimes
it's weird out here (even though Postfix has been doing this for a decade or more).
\end{aside}


\section{Model-View-Controller}

When you hear Model-View-Controller, you think about web applications.  This is a design pattern
where you place different concerns into different parts of your system and try not to mix them
too much.  For an interactive application, if you keep the part that stores data (Model) separated
from the logic (Controller) and use another piece to display and interact with the user (View), then
it's easier to change the system and adapt it over time to new features.

The power of MVC is simply that these things really are separate orthogonal pieces that get
ugly if they're mixed together.  There's no math or theory that says why; just lots of
experience has told us it's usually a bad idea.  When you start mixing them, you find out that
it's hard to change for new requirements later, because you've sprinkled logic all over your
web pages.  Or you can't update your database because there's all these stored procedures that
assume the tables are a certain way.

Mongrel2 needed a way to allow you to use various languages and tools to automate its configuration.
Letting you automate your deployments is the entire point of the server.  The idea was that if we
gave you the Controller and the Model, then you can craft any \emph{View} you wanted, and there's
no better Model than a SQL database like SQLite:  it's embeddable, easily accessed from C or any
language, portable, small, fast enough and full of all the features you need and then some.

What you are doing when you use m2sh to configure a configuration for Mongrel2, is working with
a View we've given you to create a Model for the Mongrel2 server to work with.  That's it, and
you can create your own View if you want.  It could be automated deployment scripts, a web
interface, monitoring scripts, anything you need.

The point is, if you just want to get Mongrel2 up and running, then use m2sh.  If you want to
do more advanced stuff, then get into the configuration database schema and see what you can
do.  The structure of the database very closely matches Mongrel2's internal structure, so
understanding that means you understand how Mongrel2 works.  This is a vast improvement over
other web servers like Apache where you've got no idea why one stanza has to go in a particular
place, or why information has to be duplicated.

With Mongrel2, it's all right there.


\section{Trying m2sh}

To give this configuration system a try you just need to run the test configuration used
in the unit tests.  Let's try doing a few of the most basic commands with this configuration:

\begin{code}{Sample m2sh Commands}
\begin{lstlisting}
> m2sh dump -db tests/config.sqlite
> m2sh servers -db tests/config.sqlite
> m2sh hosts -db tests/config.sqlite -host localhost
> m2sh running -db tests/config.sqlite -host localhost
> m2sh start -db tests/config.sqlite -host localhost
\end{lstlisting}
\end{code}

At this point you should have seen some raw dumps of the database, lists of servers and hosts,
seen that mongrel2 is not running, and then started it.  You can find out about all the commands
and get help for them with \shell{m2sh help} or \shell{ms2h help -for command}.

You can now try doing some simple starting, stopping and reloading using sudo (make sure you CTRL-c
to exit from the previous start command):

\begin{code}{Starting, Stopping, Reloading}
\begin{lstlisting}
> m2sh start -db tests/config.sqlite -host localhost -sudo
> tail logs/error.log
> m2sh reload -db tests/config.sqlite -host localhost
> tail logs/error.log
> curl http://localhost:6767/
> tail logs/error.log
> m2sh running -db tests/config.sqlite -host localhost
> m2sh stop -db tests/config.sqlite -host localhost
\end{lstlisting}
\end{code}

Awesome, right?  Using just this one little Python management script, you
are able to completely manage a Mongrel2 instance without having to hack
on a config file at all.  But you probably need to know how this is all
working anyway.


\subsection{What The Hell Just Happened?}

You now have done nearly everything you can to a configuration, but you might not know exactly
what's going on.  Here's an explanation of what's going on behind the scenes:

\begin{enumerate}
\item When you did \shell{m2sh start} with the \shell{-sudo} option, it actually runs
    \shell{sudo mongrel2 tests/config.sqlite localhost} to start the server.
\item Mongrel2 is now running in the background as a daemon process, just like a regular server.
    However, what it did was chroot to the current directory and then drop privileges so that
    they match the owner of that directory (you).  Use \shell{ps aux} to take a look.
\item With Mongrel2 running, you can look in the logs/error.log file to see what it said.  It should
    be a bunch of debug logging, but check out the messages: nice and detailed.
\item Next you did a soft reload with \shell{m2sh reload} and you should notice that your mongrel2
    process was able to load the new config \emph{without restarting}.
\item However, there's a slight bug that doesn't do the reload until the next request is served. That's
    what the \shell{curl http://localhost:6767/} was for.
\item Now that you can see this reload work in \file{logs/error.log}, you used \shell{m2sh running} to
    see if it's running.  This command is just reading the config database to find out where the PID file
    is (\file{run/mongrel2.pid}) and then checking if that process is running.
\item Finally, you tell mongrel2 to stop, and since it dropped privileges to be owned by you, you can do
    that without having to use sudo.
\end{enumerate}

All of this is happening by reading the \file{tests/config.sqlite} file and not reading any configuration
files.  You can now try building your own configuration that matches this one or some others.


\section{A Simple Configuration File}

To configure a new config database you'll write a Python file that looks a lot like
a configuration file.  The advantage of using Python is that you can put actual
real logic in your configuration file in order to make your configuration smarter.
A great application of this is when your configuration has to change depending on the
server it is deployed on, but you want to check the configuration file into a revision
control system.  With Python you just write your configuration like it's code and
you're set.

The first thing you need to do is initialize a fresh configuration database to get
started by using \shell{m2sh init} and then you load your configuration into it using
\shell{m2sh load}.  For our example, we'll use the example configuration
from \file{examples/python/tests/sample\_conf.py} to make a simple one:

\begin{code}{Simple Little Config Example}
  \lstinputlisting[language=Python]{../../examples/python/tests/sample_conf.py}
\end{code}

If you aren't familiar with Python then this code might look freaky, but it's really
simple.  We'll get into how its structured in a second, but to load this file
we would just do this:

\begin{code}{Loading The Simple Config}
\begin{lstlisting}
> m2sh init -db simple.sqlite
> m2sh load -db simple.sqlite -config examples/python/tests/sample_conf.py
> m2sh servers -db simple.sqlite
> m2sh hosts -db simple.sqlite -host localhost
> m2sh start -db simple.sqlite -host localhost
\end{lstlisting}
\end{code}

With this sequence of commands you:

\begin{enumerate}
\item Create a raw fresh config database name \file{simple.sqlite}
\item Load the \file{sample\_conf.py} into it.
\item List the servers it has configured.
\item List the hosts that server has, with what routes it has.
\item Start this server to try it out.
\end{enumerate}

By now you should be getting the hang of the pattern here, which is to use
m2sh and little Python configuration ``script'' to generate .sqlite files
that Mongrel2 understands.

\section{How A Config Is Structured}

The base structure of a Mongrel2 configuration is:

\begin{description}
  \item[Server] This is the root of a config, and you can have multiples of these in one database,
    even though each start command only runs one at a time.
  \begin{description}
    \item[Host] Servers have Hosts inside them, which say what DNS hostname Mongrel2 should answer for.
      You can have multiples of these in each Server.
    \begin{description}
      \item[Route] Hosts have Routes in them, which tells Mongrel2 what to do with URL paths and patterns
        that match them.  Routes then have \ident{Dir}, \ident{Handler} or \ident{Proxy} items in them.
      \begin{description}
        \item[Dir] A Dir serves files out of a directory, full with 304 and ETag support, default content types,
          and most of the things you need to serve them.
        \item[Proxy] A Proxy takes requests matching the Route they're attached to and sends them to another
          HTTP server somewhere else.  Mongrel2 will then act as a full proxy and also try to keep connections
          open in keep-alive mode if the browser supports it.
        \item[Handler] A Handler is the best part of Mongrel2.  It takes HTTP requests, and turns them into
          nicely packed and processed ZeroMQ messages for your asynchronous handlers.
      \end{description}
    \end{description}
  \end{description}
\end{description}

Each of these nested ``objects'' then has a set of attributes you can use to configure them, and
most of them have reasonable defaults.


\subsection{Server}

The server is all about telling Mongrel2 where to listen on its port, where to chroot,
and general server specific deployment gear.

\begin{description}
\item[uuid] A UUID is used to make sure that each deployed server is unique in your infrastructure.
    You could easily use any string that's letters, numbers, or - characters.
\item[chroot] This is the directory that Mongrel2 should chroot to and drop privileges.
\item[access\_log] The access log file \emph{relative to the chroot}.  Usually starts with a `/'.  Make sure
    you don't configure your server so that this and other files are accessible, or make this owned by root.
\item[error\_log] The error log file just like access\_log.
\item[pid\_file] Like the access log, where within the chroot directory is the pid file stored.
\item[default\_host] The server has a bunch of hosts listed, but it needs to know what the default host is.  This is also
    used as a convenient way to refer to this Server.
\item[port] The port the server should listen on for new connections.
\end{description}


\subsection{Host}

A host is matched using a kind of \emph{inverse route} that matches the ending of \ident{Host:}
headers against a pattern.  You'll see how this works when we talk about routes, but for now
you just need to know that request to the \ident{Server.port} are routed based on these \ident{Host}
configurations the \ident{Server} contains.

\begin{description}
\item[name] The name that you use to talk about this Host in the server configuration.
\item[matching] This is a pattern that's used to match incoming Host headers for routing purposes.
\item[server] If you want to set the server separately you can use this attribute.
\item[maintenance] This will a setting for the future that will let you have Mongrel2 throw up a maintenance page
    for this host.
\item[routes] This is a dict (hashmap) of the URL patterns mapped to the targets that should be run.
\end{description}


\subsection{Route}

The \ident{Route} is the workhorse of the whole system.  It uses some very fancy but still simple
code in Mongrel2 to translate \ident{Host:} headers to \ident{Hosts} and URL paths to \ident{Handlers},
\ident{Dirs}, and \ident{Proxies}.

\begin{description}
\item[path] This is path \emph{pattern} that matches a route.  The pattern uses the Mongrel2 pattern langauge
    which is a reduced version of the Lua pattern matching system.
\item[reversed] Determines if this pattern is reversed, which is useful for matching file extensions, hostnames,
    and other naming systems where the ending is really the prefix.  Usually you don't set this.
\item[host] You can use this attribute to set the host manually.
\item[target] This is the target that should handle the request, either a \ident{Dir}, \ident{Handler} or \ident{Proxy}.
\end{description}

Later on, you'll learn about the pattern matching that's used, but it's basically a stripped down version of your
normal regular expressions, but with a few convenient syntaxes for doing simple string matching.  When you configure
a route, you write something like \file{/images/(.*.jpg)} and the part before the `(' is used as a fast matched
prefix, while the part after it is considered a pattern to match.  When a request comes in, Mongrel2 quickly finds the
longest prefix that matches the URL, and then tests its pattern if there is one.  If the pattern is valid, the
request goes through.  If not, 404.


\subsection{Dir}

A \ident{Dir} is a simple directory-serving route target that serves files out of a directory.  It has caching
built-in, handles if-modified-since, ETags, and all the various bizarre HTTP caching mechanisms as RFC-accurate
as possible.  It also has default content-types and index files.

\begin{description}
\item[base]  This is the base directory \emph{from the chroot} that is served.  Files should not
    be served outside of this base directory even in the chroot.
\item[index\_file] This is the default index file to use if a request doesn't give one.  The \ident{Dir}
    also will do redirects if a request for a directory doesn't end in a / slash.
\item[default\_ctype] The default Content-Type to use if none matches the MIMEType table.
\end{description}

Currently, we don't offer more parameters for configuration, but eventually you'll be able to tweak more and
more of the settings to control how Dirs work.

\subsection{Proxy}

A Proxy is a Ghetto.  It is used so that you can use Mongrel2 but not have to throw
out your existing infrastructure.  Mongrel2 goes to great pains to make sure that it
implements a fast and dead accurate proxy system internally, but no matter how good it
is it, can't compete with ZeroMQ handlers.  The idea with giving Proxy functionality is
you can point Mongrel2 at existing servers, and then slowly carve out pieces that
will work as handlers.


\begin{description}
\item[addr] The DNS address of the server.
\item[port] The port to connect to.
\end{description}

Requests that match a Proxy route are still parsed by Mongrel2's incredibly accurate
HTTP parser, so that your backend servers should not be receiving badly formatted
HTTP requests.  Responses from a Proxy server, however, are sent unaltered to the
browser directly.


\subsection{Handler}

Now we get to the best part: the ZeroMQ \ident{Handlers} that will receive asynchronous requests
from Mongrel2.  You need to use the ZeroMQ syntax for configuring them, but this means with one
configuration format you can use handlers that are using UDP, TCP, Unix, or PGM transports.  Most
testing has been done with TCP transports.

\begin{description}
\item[send\_spec]  This is the 0MQ sender specification, something like ``tcp://127.0.0.1:9999'' will
    use TCP to connect to a server on 127.0.0.1 at port 9999.  The type of socket used is a DOWNSTREAM
    socket, so that handlers receive messages in round-robin style.
\item[send\_ident] This is an identifier (usually a UUID) that will be used to register the send
    socket.  This makes it so that messages are persisted between crashes.
\item[recv\_spec] Same as the send spec, but it's for receiving responses from Handlers.  The type of
    socket used is a SUB socket, so that a cluster of Mongrel2 servers will receive handler responses
    but only the one with the right recv\_ident will process it.
\item[recv\_ident] This is another UUID if you want the receive socket to subscribe to its messages.
    Handlers properly mention the send\_ident on all returned messages, so you should either set this
    to nothing and don't subscribe, or set it to the same as send\_ident.
\end{description}

The interesting thing about the \ident{Handler} configuration is you don't have to say where the
actual backend handlers live.  Did you notice you aren't declaring large clusters of proxies, proxy selection
methods, or anything else, other than two 0MQ endpoints and some identifiers?  This is because Mongrel2 is
\emph{binding} these sockets and listening.  Mongrel2 doesn't actively connect to backends; they connect
to Mongrel2.  This means, if you want to fire up 10 more handlers, you just start them; no need to restart
or reconfigure Mongrel2 to make them active.


\subsection{Others}

There's also \ident{Log}, \ident{MIMEType}, and \ident{Setting} objects/tables you can work
with, but we'll get into those later since you don't need to know about them to understand
the Mongrel2 structure.



\section{A More Complex Example}

All of this knowledge about the Mongrel2 configuration structure can now be used to take a look at
a more complex example.  We'll take a look at this example and I'll just say what's
going on, and you try to match what I'm saying to the code.  Here's the file \file{examples/python/tests/mongrel2\_org.py}:

\begin{code}{Mongrel2.org Config Script}
  \lstinputlisting[language=Python]{../../examples/python/tests/mongrel2_org.py}
\end{code}

If you haven't guessed yet, this configuration is what's used on \url{http://mongrel2.org}
to configure the main test system.  In it we've got the following things to check out:

\begin{enumerate}
\item Our basic server, with a default host of mongrel2.org.
\item The route targets are separated out into their own variables, unlike the \file{sample\_conf.py} file
    where they're just tossed into one big structure.
\item First target is a \ident{Dir} that serves up files out of the \file{tests} directory and uses \file{index.html}
    as its default file.
\item Next we setup a \ident{Proxy} pointing at the main website's server for testing the proxy.
\item Then there's a \ident{Dir} target for the \url{http://mongrel2.org:6767/chatdemo/} that we'll look at later.
\item And you have the \ident{Handler} for the same chat demo that does the actual logic of a chat system.
\item After that's a little \ident{Handler} for testing out doing HTTP requests to a handler.  Notice how even
    though the chat demo and this handler use different protocols (chat demo is using JSSockets) you don't have
    tell mongrel2 that?  It figures it out based on how they're being used rather than by configurations.
\item With all those handler targets, we can now make the \ident{mongrel2} \ident{Host} with all the routes
    assigned once, nice and clean.  However, look how I was lazy and just tossed the mp3stream demo
    right into the routes dict?  You can totally do this and m2sh will figure it out.  Remember also that
    you can use the \verb|r'blah'| string format to not have to double up on your \verb|\| chars in the patterns.
\item We then assign this \ident{mongrel2} variable as the hosts for the \ident{main} server.
\item There is also a \ident{settings} feature, which is just a dict of global settings you can tweak.  In this case,
    we're upping the number of threads that 0MQ is using for its operations.
\item Finally, we commit the whole thing to the database by passing in the servers to save and the settings
    to use.
\end{enumerate}

And that, my friends, is the most complex configuration we have so far.


\section{Routing And Host Patterns}

The pattern code was taken from \href{http://www.lua.org/}{Lua} and is some of
the simplest code for doing fast pattern matches.  It is very much like regular
expressions, except it removes a lot of features you don't need for routes.
Also, unlike regular expressions, URL patterns always match from the start.
Mongrel2 uses them by breaking routes up into a prefix and pattern part.  It
then uses routes to find the longest matching prefix and then tests the
pattern.  If the pattern matches, then the route works.  If the route doesn't
have a pattern, then it's assumed to match, and you're done.

The only caveat is you have to wrap your pattern parts in parenthesis, but these don't mean anything
other than to delimit where a pattern starts.  So instead of \shell{/images/.*.jpg}, write \shell{/images/(.*.jpg)}
for it to work.

Here's the list of characters you can use in your patterns:

\begin{itemize}
\item \verb|.| (period) All characters.
\item \verb|\a| Letters.
\item \verb|\c| Control characters.
\item \verb|\d| Digits.
\item \verb|\l| Lowercase letters.
\item \verb|\p| Punctuation characters.
\item \verb|\s| Space characters.
\item \verb|\u| Uppercase letters.
\item \verb|\w| Alphanumeric characters.
\item \verb|\x| Hexadecimal digits.
\item \verb|\z| The 0 character (null terminator).
\item \verb|[set]| Just like a regex [] where is a set of chars, like [0-9] for all digits.
\item \verb|[^set]| Inverse character set, so [\^0-9] is anything but digits.
\item \verb|*| Longest match of 0 or more of the preceding character.
\item \verb|+| Longest match of 1 or more of the preceding character.
\item \verb|-| Shortest match of 0 or more of the preceding character.
\item \verb|?| 0 or 1 match of of the preceding character
\item \verb|\b|\emph{xy} Balanced match a substring starting with \emph{x}
  and ending in \emph{y}.  So \verb|\b()| will match balanced parentheses.
\item \verb|$| End of the string.
\end{itemize}

Using the uppercase version of an escaped character makes it work the opposite
way (i.e., \verb|\A| matches any character that \emph{isn't} a letter). The backslash
can be used to escape the following character, disabling its special abilities (i.e.,
\verb|\\| will match a backslash).

Anything that's not listed here is matched literally.  Remember also that when you write the Python
code for your scripts you can use the \verb|r'blah'| string syntax to avoid having to double up on your
\verb|\| characters in the string.

\begin{aside}{Sorry, Unicodians, It's All ASCII}
Yep, I get it.  You think that everyone should use UTF-8 or some Unicode encoding for everything.
You despise the dominance of the `A' in ASCII and hate that you can't put your spoken language
right in a URL.

Well, I hate to say it, but tough.  Protocols are hard enough without having to
worry about the bewildering mess that is Unicode.  When you sit down to write a
network protocol, the last thing you need is a format that's inconsistent, has
multiple interpretations, can't be properly capitalized or lowercased, and
requires extra translations steps for every operation.  With ASCII, every
computer just knows what it is, and it's the fastest for creating wire protocol
formats.

This is why, on the Internet, you have to do things to URLs to make them ASCII,
like encoding them with \% signs.  It's in the standard, and it's the smart
thing to do.  I don't want to have to know the difference between the various
accents in your spoken language to route a URL around.  I just want to deal
with a fixed set of characters and be done with it.  Don't blame me or Mongrel2
for this, it's just the way the standard is and the way to get a server that is
stable and works.

Protocols work better when there's less politics in their design.  This means
you can't put Unicode into your URL patterns.  I mean, you can try; but the
behavior is completely undefined.

\end{aside}


Here are some example routes you can try to get a feel for the system:

\begin{itemize}
\item \verb|r"/images/"|  This will just match any path that has /images/ in it without any patterns.
\item \verb|r"/"| The fastest possible route you can have.
\item \verb|r"/images/(.*.jpg)"| Match only requests for jpg images in the images directory.  Keep in mind that this
    isn't actually looking in the directory, it's just matching the \ident{(.*.jpg)} pattern.
\item \verb|r"/images/(\a-\-\d+\.jpg)"| A more complex example that matches a short sequence of 0 or more letters (remember -), then a dash
    (\verb|\-| escapes the -), then 1 or long sequence of digits and finally a .jpg) with the \verb|\|. escaping the period.
\end{itemize}

That should give the idea of how you can use them.  Notice also that I'm using the Python \verb|r"blah"| string syntax which is
interchangeable with the \verb|r'blah'| syntax so I don't have to double escape everything.


\section{Deployment Logs And Commits}

A very nice feature for people doing operations work is that m2sh keeps track of all the
commands you run on it while you work, and lets you add little commit logs to the
log for documentation later.  These commit logs are then maintained even across
\shell{m2sh init} and \shell{m2sh load} commands so you can see what's going on.  They
track who did something, what server they did it on, what time they did it and what they
did.

To see the logs for your own tests, just do \shell{m2sh log -db simple.sqlite} and then,
if you want to add a commit log message, you use the \shell{m2sh commit} command.
Here's an example from mongrel2.org:

\begin{code}{Example Commit Log}
\begin{lstlisting}
> m2sh log -db config.sqlite 
[2010-07-18T04:14:53, mongrel2@zedshaw, init_command] /usr/bin/m2sh init -db config.sqlite
[2010-07-18T04:15:06, mongrel2@zedshaw, load_command] /usr/bin/m2sh load -db config.sqlite -config examples/python/tests/mongrel2_org.py
[2010-07-18T04:22:06, mongrel2@zedshaw, load_command] /usr/bin/m2sh load -db config.sqlite -config examples/python/tests/mongrel2_org.py
[2010-07-18T04:23:32, mongrel2@zedshaw, load_command] /usr/bin/m2sh load -db config.sqlite -config examples/python/tests/mongrel2_org.py
[2010-07-18T04:26:16, mongrel2@zedshaw, upgrade] Latest code for Mongrel2.
[2010-07-18T18:05:59, mongrel2@zedshaw, load_command] /usr/bin/m2sh load -db config.sqlite -config examples/python/tests/mongrel2_org.py
[2010-07-18T20:09:01, mongrel2@zedshaw, init_command] /usr/bin/m2sh config -db config.sqlite -config examples/python/tests/mongrel2_org.py
[2010-07-18T20:09:02, mongrel2@zedshaw, load_command] /usr/bin/m2sh config -db config.sqlite -config examples/python/tests/mongrel2_org.py
> m2sh commit -db config.sqlite -what mongrel2.org -why "Testing things out."
\end{lstlisting}
\end{code}

The motivation for this feature is the trend that operations store server configurations
in revision control systems like git or etckeeper.  This works great for holding the configuration
files, but it doesn't tell you what happened on each server.  In many cases, the configuration
files also need to be reworked or altered for each deployment.  With the m2sh log and commits
system you can augment your revision control with deployment action tracking.

Later versions of Mongrel2 will keep small amounts of statistics which will link these actions
to changes in Mongrel2 behavior like frequent crashing, failures, slowness, or other problems.

Basically, there's nowhere to hide.  Mongrel2 will help operations figure out who needs to
get fired the next time Twitter goes down.


\section{Tweakable Expert Settings}

Many of Mongrel2's internal settings are configurable using the settings system.
Some of these are dangerous to mess with, so make sure you test any changes before
you try to run them.  Setting them to 0 or negative numbers isn't checked, so if
you make a setting and things go crazy, you need to not make that setting.  All
of these have good defaults so you can leave them alone unless you need to change
them.

To configure your settings you just pass them to the \ident{commit} command when
you setup your configuration like this:

\begin{code}{Changing Settings}
\begin{lstlisting}
settings = {"zeromq.threads": 1, "limits.url_path": 1024}

commit([main], settings=settings)
\end{lstlisting}
\end{code}

Mongrel2 will read these on the fly and write INFO log messages telling you
what the settings are so you can debug them if they cause problems.  The list
of available settings are:

\begin{description}
\item[limits.buffer\_size=2 * 1024] Internal IO buffers used for things like proxying and handling requests.  This is a \emph{very} conservative setting, so if you head HTTP headers greater than this, you'll want to increase this setting.  You'll also want to shoto whoever is sending you those requests because the average is 400-600 bytes.
\item[limits.connection\_stack\_size=32 * 1024] Size of the stack used for connection coroutines.  If you're trying to cram a ton of connections into very little ram, see how low this can go.
\item[limits.content\_length=20 * 1024] Maximum allowed content length on submitted requests.  This is right now a hard limit so requests are rejected that go over it.  Later versions of Mongrel2 will use an upload mechanism that will allow any size upload.
\item[limits.dir\_max\_path=256] Max path you can set for Dir handlers.
\item[limits.dir\_send\_buffer=16 * 1024] Maximum buffer used for file sending when we need to use one.
\item[limits.fdtask\_stack=100 * 1024] Stack frame size for the main IO reactor task.  There's only one so set it high if you can, but it could possibly go lower.
\item[limits.handler\_stack=100 * 1024] The stack frame size for any Handler tasks, probably want this high, since there's not many of these, but adjust and see what your system can support.
\item[limits.handler\_targets=128] The maximum number of connection IDs a message from a Handler may target.  It's not smart to set this really high.
\item[limits.header\_count=128 * 10] Max number of allowed headers from a client connection.
\item[limits.host\_name=256] Max hostname for Host specifiers and other DNS related settings.
\item[limits.mime\_ext\_len=128] Maximum length of MIME type extensions.
\item[limits.url\_path=256] Max URL paths, does not include query string, just path.
\item[status\_file=/tmp/mongrel2\_status.txt] When you run as root this will put it in your CHROOT/tmp/ directory, but when you run as you it puts it in the /tmp.  Change this to something better if you run multiple servers.
\item[superpoll.hot\_dividend=4] Ratio of the total (like 1/4th, 1/8th) that should be in the hot selection.  Set this higher if you have lots of idle connections, set it lower if you have more active connections.
\item[superpoll.max\_fd=10 * 1024] Maximum possible open files.  Do not set this above 64 * 1024, and expect it to take a bit while Mongrel2 sets up constant structures.
\item[upload.temp\_store=None] This is not set by default.  If you want large requests to reach your handlers, then set this to a directory they can access the results, and make sure they can handle it.  Read about it in the Hacking section under Uploads.  The file has to end in XXXXXX chars to work (read man mkstemp).
\item[zeromq.threads=1] Number of 0MQ IO threads to run.  Careful, we've experienced thread bugs in 0MQ sometimes with high numbers of these.
\end{description}


